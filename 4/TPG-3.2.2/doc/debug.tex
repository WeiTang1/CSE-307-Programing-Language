\section{Introduction}          \label{debug}

When I need to debug a grammar I often add print statments to visualize the parser activity.
Now with TPG 3 it is possible to print such information automatically.

\section{Verbose parsers}

Normal parsers inherit from \emph{tpg.Parser}.
If you need a more verbose parser you can use \emph{tpg.VerboseParser} instead.
This parser prints information about the current token each time the lexer is called.
The debugging information has currently two levels of details.

\begin{description}
    \item [Level 0] displays no information.
    \item [Level 1] displays tokens only when the current token matches the expected token.
    \item [Level 2] displays tokens if the current token matches or not the expected token.
\end{description}

The level is defined by the attribute \emph{verbose}. Its default value is 1.

\begin{code}
\caption{Verbose parser example}                            \label{debug:example}
\begin{verbatimtab}[4]
class Test(tpg.VerboseParser):
    r"""

    START -> 'x' 'y' 'z' ;

    """

    verbose = 2
\end{verbatimtab}
\end{code}

The information displayed by verbose parsers has the following format:
\begin{verbatim}
[eat counter][stack depth]callers: (line,column) <current token> == <expected token>
\end{verbatim}

\begin{description}
    \item [$eat counter$] is the number of calls of the lexer.
    \item [$stack depth$] is the depth of the Python stack since the axiom.
    \item [$callers$] is the list of non terminal symbols called before the current symbol.
    \item [$(line,column)$] is the position of the current token in the input string.
    \item [$==$] means the current token matches the expected token (level 1 or 2).
    \item [$!=$] means the current token doesn't match the expected token (level 2).
\end{description}
