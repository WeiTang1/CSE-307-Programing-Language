\section{Incremental Table Maintenance} \label{sec:incremental_tabling}
%====================================================

\index{tabling!incremental}

XSB allows the user to declare that the system should incrementally
maintain particular tables.  A table $T$ is {\em incremental} if XSB
ensures that its answers are consistent with all dynamic facts and
rules upon which $T$ depends (subject to transactionality conditions
explained below).  After a database update or series of updates
$\Delta$, a table $T$ that depends on $\Delta$ may be updated
\begin{itemize}
\item {\em Eagerly}: by issuing a command to update all tables that
  depend on $\Delta$; or
%
\item {\em Lazily}: by updating $T$ and all tables upon which $T$
  depends the next time that $T$ is called.
\end{itemize}
%
In either case, if tables are thought of as database views, then the
table maintenznce subsystem enables what is known in the database
community as incremental view maintenance~\footnote{In the current
  version of XSB, there are certain restrictions on how incremental
  tabling can be used.}.

\subsection{Examples} \label{sec:incr_examples}

To demonstrate incremental table maintenance (informally called {\em
  incremental tabling}), we consider first the following simple
program that does not use incremental tabling:
\begin{verbatim}
:- table p/2.
p(X,Y) :- q(X,Y),Y =< 5.

:- dynamic q/2.
q(a,1).    q(b,3).   q(c,5).    q(d,7).
\end{verbatim}
and the following queries and results:
\begin{verbatim}
| ?- p(X,Y),writeln([X,Y]),fail.
[c,5]
[b,3]
[a,1]

no
| ?- assert(q(d,4)).

yes
| ?- p(X,Y),writeln([X,Y]),fail.
[c,5]
[b,3]
[a,1]

no
\end{verbatim}
%
Here we see that the table for {\tt p/2} depends on the contents of
the dynamic predicate {\tt q/2}.  We first evaluate a query, {\tt
  p(X,Y)}, which creates a table.  Then we use {\tt assert/1} to add a
fact to the {\tt q/2} predicate and re-evaluate the query.  We see
that the answers haven't changed, and this is because the table is
already created and the second query just retrieves answers directly
from that existing table.  But in this case we have answers that are
inconsistent with the current definition of {\tt p/2}.  I.e., if the
table didn't exist (e.g. if {\tt p/2} weren't tabled), we would get a
different answer to our {\tt p(X,Y)} query, this time including the
{\tt [d,4]} answer.  Without incremental table maintenance, the only
solution to this problem is for the XSB programmer to explicitly
abolish a table whenever changing (with assert or retract) a predicate
on which the table depends.

By declaring that the tables for {\tt p/2} should be incrementally
maintained, XSB will automatically keep the tables for {\tt p/2}
correct.

\paragraph{A Quasi Forward-Chaining Approach}
Consider a slight rewrite of the above program:
\begin{verbatim}
:- table p/2 as incremental.
p(X,Y) :- q(X,Y),Y =< 5.

:- dynamic q/2 as incremental.
q(a,1).    q(b,3).    q(c,5).     q(d,7).
\end{verbatim}
in which {\tt p/2} is declared to be incrementally tabled (with {\tt
  :- table p/2 as incremental}) and {\tt q/2} is declared to be both
dynamic and incremental, meaning that an incremental table depends on
it~\footnote{The declarations {\tt use\_incremental\_tabling/1} and
  {\tt use\_incremental\_dynamic/1} are deprecated from \version{} of
  XSB forward -- in other words backwards compatability will be
  maintained for a time, but these declarations will not be further
  supported.}.  Consider the following goals and execution:
\begin{verbatim}
| ?- import incr_assert/1 from increval.

yes
| ?- p(X,Y),writeln([X,Y]),fail.
[c,5]
[b,3]
[a,1]

no
| ?- incr_assert(q(d,4)).

yes
| ?- p(X,Y),writeln([X,Y]),fail.
[d,4]
[c,5]
[b,3]
[a,1]

no
\end{verbatim}
Here again we call {\tt p(X,Y)} and generate a table for it and its
answers.  Then we update {\tt q/2} by using the incremental version of
assert, {\tt incr\_assert/1}, which was explicitly imported.  Now when
we call {\tt p(X,Y)} again, the table has been updated and we get the
correct answer.

In this case after every {\tt incr\_assert/1} and/or {\tt
  incr\_retract(all)/1}, the tables are incrementally updated to
reflect the change.  The system keeps track of what tabled goals
depend on what other tabled goals and (incremental) dynamic goals, and
tries to minimize the amount of recomputation necessary.
Incrementally tabled predicates may depend on other tabled predicates.
In this case, those tabled predicates must also be declared as
incremental (or opaque)~\footnote{An {\em opaque} predicate $P$ is
  tabled and is used in the definition of some incrementally tabled
  predicate but should not be maintained incrementally.  In this case
  the system assumes that the programmer will abolish tables for $P$
  in such a way so that re-calling it will always give semantically
  correct answers.}.  The algorithm used is described
in~\cite{SaRa05,Saha06}.

\paragraph{An Eager Updating Approach}
%
There is a more efficient way to program incremental updates when
there are several changes made to the base predicates at one time.  In
this case the {\tt incr\_assert\_inval/1} and {\tt
  incr\_retract(all)\_inval/1} predicates should be used for each
individual update.  These operations leave the dependent tables
unchanged (and thus inconsistent.)  When the updates are finished, the
user then calls {\tt incr\_table\_update/0}.  which updates all tables
that depend on any changed dynamic rules or facts.  The following
execution for our running program shows an example of this.

\begin{verbatim}
| ?- import incr_assert_inval/1, incr_table_update/0 from increval.

yes
| ?- p(X,Y),writeln([X,Y]),fail.
[c,5]
[b,3]
[a,1]

no
| ?- incr_assert_inval(q(d,4)), incr_assert_inval(q(d,1)), 

yes
| ?- incr_table_update.

| ?- p(X,Y),writeln([X,Y]),fail.
[d,4]
[d,1]
[c,5]
[b,3]
[a,1]

no
\end{verbatim}

\paragraph{A Lazy Updating Approach}
%
But what if a user forgets to call {\tt incr\_table\_update/0}?  Or,
what if it is important that some tables be updated, while others can
remain inconsistent?  Beginning with Version 3.3.7 of XSB, {\em lazy
  incremental table maintenance} is supported.  In this case if a
query $Q$ is called to a table that is marked to be inconsistent with
the underlying dynamic code, but that table has not been updated, $Q$
will be dynamically updated along with all tables upon which $Q$
depends -- and only those tables will be updated.  In this manner, if
incremental table maintenance is used, inconsistent answers to a query
$Q$ will never be returned.  However, if table inspection predicates
are used (cf. Chapter~\ref{sec:TablingPredicates}) inconsistent
answers may be shown.  We note, however, that the use of these
predicates is much less common than direct queries to tables; but if
using table inspection predicates on incrementally maintained tables,
the user should ensure that {\tt incr\_table\_update/0} is called
before inspecting the tables.

\paragraph{Updating conditional answers}
%
As discussed earlier in this chapter, answers that are undefined in
the well-founded semantics are represented as conditional answers.
Beginning with version 3.3.7, incremental updates work correctly with
conditional answers (before this version they only worked correctly on
stratified tables: those with only unconditional answers).  No special
care needs to be taken for conditional answers, and they can be
updated through any of the previously described methods.  The
following example illustrates one such approach.  

Consider the program
%
\begin{verbatim}
:- dynamic data/1 as incremental.

:- table opaque_undef/0 as opaque.
opaque_undef:- tnot(opaque_undef).

:- table p/1 as incremental.
p(_X):- opaque_undef.
p(X):- data(X).
\end{verbatim}
%
Note that {\tt opaque\_undef/1} upon which {\tt p/1} depends is
explicitly declared as opaque.  When the above program is loaded, XSB
will behave as follows.
%
{\small
\begin{verbatim}
| ?- p1(1).

undefined
| ?- incr_assert(data(1)).

yes
| ?- p1(1).

yes
| ?- incr_retract(data(1)).

yes
| ?- p1(1).

undefined
| ?- get_residual(p1(1),C).

C = [opaque_undef]
\end{verbatim}
}
%



\subsubsection{Declaring Predicates to be Incremental}
%
In XSB, tables can have numerous properties: {\em subsumptive,
  variant, incremental, opaque, dynamic, private}, and {\em shared},
and can use answer subsumption or call abstraction.  XSB also has
variations in forms of dynamic predicates: {\em tabled, incremental,
  private}, and {\em shared}.  XSB extends the {\tt table} and {\tt
  dynamic} compiler and executable directives with modifiers that
allow users to indicate the kind of tabled or dynamic predicate they
want.  For example,
%
\begin{verbatim}
:- table p/3,s/1 as subsumptive,private.

:- table q/3 as incremental,variant.

:- dynamic r/2,t/1 as incremental.
\end{verbatim}
We note that
\begin{verbatim}
:- table p/3 as dyn.
and
:- dynamic p/3 as tabled.
\end{verbatim}
are equivalent.

In the current version of XSB, many combinations involving incremental
tabling are not supported and will throw an error (cf. page
\pageref{table-declaration} and page \pageref{dynamic-declaration},
respectively). Incremental tabling has not yet been ported to the
multi-threaded engine and and it currently works only for predicates
that use both call and answer variance.

\index{tries!and incremental tabling}
\subsection{Incremental Tabling using Interned Tries} \label{sec:incr-update-tries}
%
Sometimes it is more convenient or efficient to maintain facts in
interned tries rather than as dynamically asserted facts
(cf. Chapter~\ref{chap:tries}).  Tables based on interned tries can be
automatically updated when terms are interned or uninterned just as
they can be automatically updated when a fact is asserted or
retracted.  Consider the example from Section~\ref{sec:incr_examples}
rewritten to use interned tries.  As usual, an incrementally updated
table is declared as such:
%
\begin{verbatim}
:- table p/2 as incremental.
p(X,Y) :- trie_interned(q(X,Y),inctrie),Y =< 5.
\end{verbatim}
%
However, the declaration for dynamic data changes: rather than using
the declaration {\tt :- dynamic q/2 as incremental}, a trie is
specified as incremental in its creation.
%
\begin{verbatim}
  trie_create(Trie_handle,[incremental,alias(inctrie)]),
\end{verbatim}
%
As described in Chapter~\ref{chap:tries}, the trie handle returned is
an integer, but can be aliased just as with any other trie.  The trie
may then be initially loaded:
%
\begin{verbatim}
	trie_intern(q(a,1),inctrie),trie_intern(q(b,3),inctrie),
	trie_intern(q(c,5),inctrie),trie_intern(q(d,7),inctrie).
\end{verbatim}
%
At this stage a query to {\tt p/2} acts as before:
%
\begin{verbatim}
| ?- p(X,Y),writeln([X,Y]),fail.
[c,5]
[b,3]
[a,1]
\end{verbatim}
%
The following sequence ensures that {\tt p/2} is incrementally updated
as {\tt inctrie} changes:
%
\begin{verbatim}
| ?- import incr_trie_intern/2.

yes
| ?- incr_trie_intern(inctrie,q(d,4)).

yes
| ?- p(X,Y),writeln([X,Y]),fail.
[d,4]
[c,5]
[b,3]
[a,1]

no
\end{verbatim}
%
There are also {\tt incr\_trie\_intern\_inval/2}, and {\tt
  incr\_trie\_unintern\_inval/2} predicates which do not immediately
update dependent tables.  Lazy incremental table maintenance works for
changes made to interned tries just as it does for regular dynamic
code and for trie-indexed dynamic code.


\subsection{View Consistency} \label{sec:view-consistency}
%
In addition to the success continuations that are standard in most
languages, Prolog has failure continuations -- choice points to take
upon backtracking.  The presence of these failure continuations leads
to an issue of view consistency, even within a single-threaded
computation.  Suppose that a user
%
\begin{enumerate}
\item Makes a query to a completed incrementally tabled subgoal $Q$.
  $Q$ has more than one solution and the first one is returned,
  leaving a choice point into the table for $Q$.
\item Makes an update to dynamic code upon which $Q$ depends
\item Makes another query to $Q$
\end{enumerate}
%
What is the relation between the queries and the update.  Presumably,
the first query in step 1) should not reflect the changes made in step
2) if a user backtracks for further answers to that query -- this can
be seen as ensuring view consistency.. However it is less clear
whether the second query to $Q$ in step 4) should return the same
answers as the first query in step 1), or whether the second query
should reflect the database update.  Arguments can be made for either
approach.
%
\begin{itemize}
\item {\em Prolog-style semantics} If the second query reflects the
  database change, it is consistent with the database, but is not
  consistent with the first query~\footnote{This approach could be
    viewed as an extension of the ISO semantics for dynamic code in
    Prolog, which XSB does not currently support.};
\item {\em Delayed update semantics} If the second query does not
  reflect the dynamic code change, it is consistent with the first
  query but not with the dynamic code change.
\end{itemize}
%
XSB chooses the latter of these approaches.  If a user has failure
continuations into a query $Q$, then $Q$ and all tables that depend on
$Q$ will not be updated until these failure continuations have been
exhausted or removed.  However, all updates are ensured to be applied
once this is the failure continuations are removed.

\subsection{Summary and Implementation Status}
%
Thus the user has four choices: tables may be updated as soon as the
database is changed (e.g., via {\tt incr\_assert/1}); at some point
after a series of database changes (e.g. via {\tt
  incr\_assert\_inval/1} and {\tt incr\_table\_update/0}); or lazily
whenever a given table is called.  In addition, if the changes are so
massive that there is no point in incrementally updating the table,
the tables can be abolished so that the tables will be reconstructed
whenever they are re-queried.

In the current version of XSB, incremental tabling has not yet been
ported to the multi-threaded engine.  In addition, incremental tabling
only works for predicates that use both call and answer variance.
However, incremental tabling does work with for the full well-founded
semantics, for trie indexed dynamic code (in addition to regular
dynamic code) and with interned tries as described in
Section~\ref{sec:incr-update-tries}.  The space reclamation predicates
{\tt abolish\_all\_tables/0} and {\tt abolish\_table\_call/[1,2]} both
can be safely used with incremental tables, but {\tt
  abolish\_table\_pred/[1,2]} if the predicate it abolishes is
incremental.

\subsection{Predicates for Incremental Table Maintenance} \label{sec:incr-preds1}

\paragraph{A Note on Terminology}
%
Suppose {\tt p/1} and {\tt q/1} are incrementally tabled, and that
there is a clause
%
\begin{verbatim}
p(X):- q(X).
\end{verbatim}
%
In this case we say that {\tt p(X)} {\em depends\_on} {\tt q(X)} and
that {\tt q(X)} {\em affects} {\tt p(X)}.  A recursive predicate both
depends on and affects itself.


\paragraph{Declarations} The following directives support incremental
tabling based on changes in dynamic code: 

\index{tabling!opaque}
\index{tabling!declarations}
\begin{description}
\index{tabling!incremental}

\ourstandarditem{table +PredSpecs as incremental}{table/1}{Tabling}
%
is a executable predicate that indicates that each tabled predicate
specified in {\tt PredSpec} is to have its tables maintained
incrementally.  {\tt PredSpec} is a list of skeletons, i.e. open
terms, or {\tt Pred/Arity} specifications~\footnote{No explicit module
  references are allowed.}.  The tables must use call variance and
answer variance and must compiled and loaded into the single-threaded
engine.  If a predicate is already declared as subsumptively tabled,
an error is thrown.  This predicate, when called as a compiler
directive, implies that its arguments are tabled predicates.  See page
\pageref{table-declaration} for further discussion of tabling options.

We also note that any tabled predicate that is called by a predicate
tabled as incremental must also be tabled as incremental or as opaque.
On the other hand, a dynamic predicate {\tt d/n} that is called by a
predicate tabled as incremental may or may not need to be declared as
incremental.  However if {\tt d/n} is not declared incremental, then
changes to it will not be propagated to incrementally maintained
tables.

\ourstandarditem{dynamic +PredSpecs as incremental}{dynamic/1}{Tabling}
%
is an executable predicate that indicates that each predicate in {\tt
  PredSpecs} is dynamic and used to define an incrementally tabled
predicate and will be updated using {\tt incr\_assert/1} and/or {\tt
  incr\_retractall/1} (or relatives.)  This predicate, when called as
a compiler directive, implies that its arguments are dynamic
predicates.  See page \pageref{dynamic-declaration} for further
discussion of dynamic options.

\ourstandarditem{table +PredSpecs as opaque}{table/1}{Tabling} 
%
is an executable predicate that indicates that each predicate $P$ in
{\tt PredSpecs} is tabled and is used in the definition of some
incrementally tabled predicate but should not be maintained
incrementally.  In this case the system assumes that the programmer
will abolish tables for $P$ in such a way so that re-calling it will
always give semantically correct answers.  In other words, instead of
maintaining information to support incremental table maintenance, the
system re-calls the opaque predicate whenever its results are required
to recompute an answer.  One example of an appropriate use of opaque
is for tabled predicates in a DCG used to parse some string.  Rather
than incrementally maintain all dependencies on all input strings, the
user can declare these intermediate tables as opaque and abolish them
before any call to the DCG.  This predicate, when called as a compiler
directive, implies that its arguments are tabled predicates.

\end{description}

\paragraph{Basic Incremental Maintenance Predicates}
The following predicates are used to manipulate incrementally
maintained tables:

\begin{description}
\ourrepeatmoditem{incr\_assert(+Clause)}{incr\_assert/1}{increval} 
\ourrepeatmoditem{incr\_assertz(+Clause)}{incr\_assertz/1}{increval}
\ourrepeatmoditem{incr\_asserta(+Clause)}{incr\_asserta/1}{increval}
\ourrepeatmoditem{incr\_retract(+Clause)}{incr\_retract/1}{increval}
\ourmoditem{incr\_retractall(+Term)}{incr\_retractall/1}{increval}
% 
are versions of {\tt assert/1} and other standard Prolog predicates.
They modify dymamic code just as their Prolog counterparts, but they
then immediately updates all incrementally maintained tables that
depend on {\tt Clause}.

{\bf Error Cases} are the same as {\tt assert<a/z>/1}, {\tt retract/1}
and {\tt retractall/1} with the additional error condition:

\begin{itemize}
\item The head of the clause {\tt Clause} or the {\tt Term} refers to
  a predicate that is not incremental and dynamic.  
\bi
\item  {\tt type error(dynamic\_incremental, Term)}
\ei
\end{itemize}

\ourrepeatmoditem{incr\_assert\_inval(+Clause)}{incr\_assert\_inval/1}{increval}
\ourrepeatmoditem{incr\_assertz\_inval(+Clause)}{incr\_assertz\_inval/1}{increval}
\ourrepeatmoditem{incr\_asserta\_inval(+Clause)}{incr\_asserta\_inval/1}{increval}\
\ourrepeatmoditem{incr\_retractall\_inval(+Clause)}{incr\_retractall\_inval/1}{increval}
\ourmoditem{incr\_retract\_inval(+Term)}{incr\_retract\_inval/1}{increval}
%
are versions of {\tt assert/1} and other standard Prolog predicates.
They modify dymamic code just as their Prolog counterparts, and mark
any incrementally maintained tables that depend on the modification as
invalid (in need of updating).  The tables may be updated by an
explicit call to {\tt incr\_table\_update/[0,1,2]}, or the table will
be dynamically recomputed when a query is made to it.

\ourmoditem{incr\_table\_update}{incr\_table\_update/0}{increval} may
be called after base predicates have been changed (by {\tt
  incr\_assert\_inval/1} and/or \linebreak {\tt
  incr\_retractall\_inval/1} or friends).  This predicate updates all
the incrementally maintained tables whose contents change as a result
of those changes to the base predicates.  This update operation is
separated from the operations that change the base predicates ({\tt
  incr\_assert\_inval/1} and {\tt incr\_retractall\_inval/1}) so that
a set of base predicate changes can be processed all at once, which
may be much more efficient that updating the tables at every base
update.  Beginning with Version 3.3.7, it is not absolutely necessary
to call this predicate, as tables will be incrementally updated upon
demand.  However, using this predicate allows a choice of incurring
the cost of update at a time other than querying an updated goal.

{\bf Error Cases}
\bi
\item A table $T$ that is to be incrementally updated is not yet
  complete.  
\bi
\item 	{\tt permission\_error(update, incomplete\_table Goal)}
\ei
\ei

\ourmoditem{incr\_table\_update(-GoalList)}{incr\_table\_update/1}{increval}
acts as {\tt incr\_table\_update/0} in its action to update the
incrementally maintained tables after changes to base predicates.  It
returns the list of goals whose tables were changed in the update
process.

\ourmoditem{incr\_table\_update(+SkelList,-GoalList)}{incr\_table\_update/2}{increval}
acts as {\tt incr\_table\_update/1} in its action to update
incrementally maintained tables after changes to base predicates.  The
first argument is a list of predicate skeletons (open terms) for
incrementally maintained tables.  The predicate returns in {\tt
  GoalList} a list of goals whose skeletons appear in {\tt SkelList}
and whose tables were changed in the update process.  So {\tt
  SkelList} acts as a filter to restrict the goals that are returned
to those of interest.  If {\tt SkelList} is a variable, all affected
goals are returned in {\tt GoalList}.

\ourmoditem{incr\_invalidate\_call(+Goal)}{incr\_invalidate\_call/1}{increval}
is used to directly invalidate a call to an incrementally maintained
table, {\tt Goal}.  A subsequent invocation of {\tt
  incr\_table\_update/[0,1,2]} will cause {\tt Goal} to be recomputed
{\em and all incrementally maintained tables that {\tt Goal} affects
  will be updated}; similarly, a call to {\tt Goal} will automatically
perform incremental updating for {\tt Goal} along with any tables that
{\tt Goal} depends on that are in need of updating.  This predicate can
be used if a tabled predicate depends on some external data and not
(only) on dynamic incremental predicates.  If, for example, an
incrementally maintained predicate depends on a relation stored in an
external relational database (perhaps accessed through the ODBC
interface), then this predicate can be used to invalidate the table
when the external relation changes.  The application programmer must
know when the external relation changes and invoke this predicate as
necessary.

{\bf Error Cases}
\bi
\item 	{\tt Goal} is tabled, but not incrementally tabled
\bi
\item 	{\tt permission\_error(invalidate,non-incremental predicate,Goal)}
\ei
\ei

\end{description}

\paragraph{Incremental Maintenance using Interned Tries}
The following predicates are used for modifying incremental tries, and
can be freely intermixed with predicates for modifying incremental
dynamic code, as well as with predicates for invalidating or updating
tables (Section~\ref{sec:incr-preds1}).

\begin{description}
\ourmoditem{incr\_trie\_intern(+TrieIdOrAlias,+Term)}{incr\_trie\_intern/2}{intern}
%
is a version of {\tt trie\_intern/2} for tries declared as
incremental.  A call to this predicate interns {\tt Term} in {\tt
  TrieIdOrAlias} and then updates all incrementally maintained tables
that depend on this trie.

\ourmoditem{incr\_trie\_uninternall(+TrieIdOrAlias,+Term)}{incr\_trie\_uninternall/2}{intern}
%
is a version of {\tt trie\_unintern/2} for tries declared as
incremental.  A call to this predicate removes all terms unifying with
{\tt Term} in {\tt TrieIdOrAlias} and then updates all incrementally
maintained tables that depend on this trie.

\ourmoditem{incr\_trie\_intern\_inval(+TrieIdOrAlias,+Term)}
{incr\_trie\_intern\_inval/2}{intern}
%
works for tries declared as incremental in a similar manner as {\tt
  incr\_trie\_intern/2} except that it does not update the
incrementally maintained tables, but only marks them as invalid. The
tables may be updated by an explicit call to {\tt
  incr\_table\_update/[0,1,2]}, or updated lazily.

\ourmoditem{incr\_trie\_uninternall\_inval(+TrieIdOrAlias,+Term)}
{incr\_trie\_uninternall\_inval/2}{intern}
%
works for tries declared as incremental in a similar manner as {\tt
  incr\_trie\_uninternall/2} except that it does not update the
incrementally maintained tables, but only marks them as invalid. The
tables may be updated by an explicit call to {\tt
  incr\_table\_update/[0,1,2]} or updated lazily.
\end{description}

\index{residual dependency graph}
\index{incremental dependency graph}
\paragraph{Introspecting Dependencies among Incremental Subgoals}
%
In order to efficiently perform incremental updates, each
incrementally tabled subgoal $S$ contains information about other
subgoals upon which $S$ directly depends or which $S$ directly
affects.  These relations form a labelled directed graph for which the
nodes are incrementally tabled subgoals present in XSB; a given
subgoal in the graph may or may not have been completed.  In addition,
there is an edge from $S_1$ to $S_2$ labelled depends (affects) if
$S_1$ directly depends on (directly affects) $S_2$.  We call this
graph the {\em incrementally tabled subgoal dependency graph}, or just
the incremental dependency graph.  The predicates in this section
allow a user to inspect properties of the dependency graph that can be
useful in debugging or profiling a computation~\footnote{The
  predicates for traversing the incremental dependency graph are
  somewhat analogous to those for traversing the residual dependency
  graph (Section~\ref{sec:table-inspection}).}.

As explained below, nodes for the dependency graph can be accessed via
the predicate {\tt is\_incremental\_subgoal/1}, while edges can be
accessed via {\tt incr\_directly\_depends/2}.  The predicates {\tt
  get\_incr\_scc/[1,2]} and {\tt get\_incr\_scc\_with\_deps/[3,4]} can
be used to efficiently materialize the dependency graph in Prolog,
including SCC information.

\begin{description}

\ourmoditem{is\_incremental\_subgoal(?Subgoal)}{is\_incremental\_subgoal/1}{increval}
%
This predicate non-deterministically unifies {\tt Subgoal} with
incrementally tabled subgoals that are currently table entries.

\ourmoditem{incr\_directly\_depends(?Goal$_1$,?Goal$_2$)}{incr\_directly\_depends/2}{increval}
accesses the dependency structures used by the incremental table
maintenance subsystem to provide information about which incremental
table calls depend on which others.  At least one of {\tt Goal$_1$}
and {\tt Goal$_2$} must be bound.
\begin{itemize}
\item If {\tt Goal$_1$} is bound, then this predicate will return in
  {\tt Goal$_2$} through backtracking the goals for all incrementally
  maintained tables on which {\tt Goal$_1$} directly depends.
\item If {\tt Goal$_2$} is bound, then it returns in {\tt Goal$_2$}
  through backtracking the goals for all incrementally maintained
  tables that {\tt Goal$_2$} directly affects -- in other words all
  goals that directly depend on {\tt Goal$_2$}.  \ei

{\bf Error Cases}
\bi
\item Neither {\tt Goal$_1$} nor {\tt Goal$_2$} is bound 
\bi
\item 	{\tt instatiation\_error}
\ei
\item {\tt Goal$_1$} and/or {\tt Goal$_2$} is bound, but is not
  incrementally tabled
\bi
\item 	{\tt table\_error}
\ei
\ei

\ourmoditem{incr\_trans\_depends(?Goal$_1$,?Goal$_2$)}{incr\_trans\_depends/2}{increval}
is similar to {\tt incr\_directly\_depends/2} except that it returns
goals according to the transitive closure of the ``directly depends''
relation.  Error conditions are the same as {\tt
  incr\_directly\_depends/2}.

\ourrepeatmoditem{get\_incr\_sccs(?SCCList)}{get\_incr\_sccs/1}{increval}
\ourrepeatmoditem{get\_incr\_sccs\_with\_deps(?SCCList,?DepList)}{get\_incr\_sccs\_with\_deps/2}{increval}
\ourrepeatmoditem{get\_incr\_sccs(+SubgoalList,?SCCList)}{get\_incr\_sccs/2}{increval}
\ourmoditem{get\_incr\_sccs\_with\_deps(+SubgoalList,?SCCList,?DepList)}{get\_incr\_sccs\_with\_deps/3}{increval}
%
Most linear algorithms for SCC detection over a graph use destructive
assignment on a stack to maintain information about the connecteness
of a component; as a result such algorithms are
difficult to write efficiently in Prolog.

{\tt get\_incr\_sccs/1} unifies {\tt SCCList} with SCC information for
the incremental dependency graph that is represented as a list whose
elements are of the form
\begin{center}
{\tt ret(Subgoal,SCC)}.
\end{center}
{\tt SCC} is a numerical index for the SCCs of Subgoal. Two subgoals
are in the same SCC iff they have the same index, however no other
dependency information can be otherwise directly inferred from the
index.  

If dependency information is also desired, {\tt
  get\_incr\_scc\_with\_dependencies/2} should be called.  In addition
to the SCC information as above, {\tt DepList} is unified with a list
of dependency terms of the form
\begin{center}
{\tt depends(SCC1,SCC2)}
\end{center}
for each pair {\tt SCC1} and {\tt SCC1} such that some subgoal with
index {\tt SCC1} directly depends on some subgoal with index {\tt
  SCC1}.

Ordinarily a user will want to see the entire dependency graph and in
such a case the predicates described above should be used.  However,
note that if the dependency graph is the result of several
indepdendent queries it may not be connected.  {\tt get\_incr\_scc/2}
takes as input a list of incremental subgoals, {\tt SubgoalList}.  For
each {\tt Subgoal} in {\tt SubgoalList}, this predicate finds the set
of subgoals connected to {\tt Subgoal} by any mixture of depends and
affects relations, unions these sets together, and finds the SCCs of
all subgoals in the unioned set.

These predicates implement Tarjan's algorithm~\cite{Tarj72} in C
working directly on XSB's data structures.  The algorithm is $\cO(|V|
+ |E|)$ where $|V|$ is the number of vertices and $|E|$ the number of
edges in the dependency graph.  As a result, these predicates provide
an efficient means to materialize the dependency graph, even if SCC
information per se is not required.

{\bf Error Cases}
\bi
\item {\tt SCCList} contains a predicate that is not tabled
\bi
\item 	{\tt permission\_error}
\ei
\ei
\end{description}

