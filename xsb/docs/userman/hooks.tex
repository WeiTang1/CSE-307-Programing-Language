\chapter{Hooks} \label{hooks}

Sometimes it is useful to let the user application catch certain
events that occur during XSB execution. For instance, when the user
asserts or retracts a clause, etc.  XSB has a general mechanism by
which the user program can register \emph{hooks} to handle certain
supported events. All the predicates described below must be imported
from {\tt xsb\_hook}.


\section{Adding and Removing Hooks}

A hook in XSB can be either a 0-ary predicate or a unary predicate.
A 0-ary hook is called without parameters and unary hooks are called with
one parameter. The nature of the parameter depends on the type of the hook,
as described in the next subsection.


\begin{description}
\ourmoditem{add\_xsb\_hook(+HookSpec)}{add\_xsb\_hook/1}{xsb\_hook}

This predicate registers a hook; it must be imported from {\tt xsb\_hook}.
{\tt HookSpec} has the following format:
%%
\begin{quote}
 {\tt
   hook-type(your-hook-predicate(\_))
   }
\end{quote}
%%
or, if it is a 0-ary hook:
%%
\begin{quote}
  {\tt
   hook-type(your-hook-predicate)
   }  
\end{quote}
%%
For instance, 
%%
\begin{verbatim}
    :- add_xsb_hook(xsb_assert_hook(foobar(_))).
\end{verbatim}
%%
registers the hook {\tt foobar/1} as a hook to be called when XSB
asserts a clause. Your program must include
clauses that define {\tt foobar/1}, or else an error will result.

The predicate that defines the hook type must be imported from {\tt
  xsb\_hook}:
%%
\begin{verbatim}
    :- import xsb_assert_hook/1 from xsb_hook.  
\end{verbatim}
%%
or {\tt add\_xsb\_hook/1} will issue an error.

\ourmoditem{remove\_xsb\_hook(+HookSpec)}{remove\_xsb\_hook/1}{xsb\_hook}

Unregisters the specified XSB hook; imported from {\tt xsb\_hook}. For
instance,
%%
\begin{verbatim}
    :- remove_xsb_hook(xsb_assert_hook(foobar(_))).
\end{verbatim}
%%
As before, the predicate that defines the hook type must be imported from
{\tt xsb\_hook}.
\end{description}


\section{Hooks Supported by XSB}

The following predicates define the hook types supported by XSB. They must
be imported from {\tt xsb\_hook}.

\begin{description}
\ourmoditem{xsb\_exit\_hook(\_)}{xsb\_exit\_hook/1}{xsb\_hook} 

These hooks are called just before XSB exits. You can register as many
hooks as you want and all of them will be called on exit (but the order of
the calls is not guaranteed). Exit hooks are all 0-ary and must be registered
as such:
%%
\begin{verbatim}
    :- add_xsb_hook(xsb_exit_hook(my_own_exit_hook)).
\end{verbatim}
%%


\ourmoditem{xsb\_assert\_hook(\_)}{xsb\_assert\_hook/1}{xsb\_hook} 
%
These hooks are called whenever the program asserts a clause. An assert
hook must be a unary predicate, which expects the clause
being asserted as a parameter. For instance,
%%
\begin{verbatim}
    :- add_xsb_hook(xsb_assert_hook(my_assert_hook(_))).
\end{verbatim}
%%
registers {\tt my\_assert\_hook/1} as an assert hook. One can register
several assert hooks and all of them will be called (but the order is not
guaranteed).

\ourmoditem{xsb\_retract\_hook(\_)}{xsb\_retract\_hook/1}{xsb\_hook} 
%
These hooks are called whenever the program retracts a clause. A retract
hook must be a unary predicate, which expects as a parameter a list of the
form {\tt [Head,Body]}, which represent the head and the body parts of the
clause being retracted. As with assert hooks, any number of retract hooks
can be registered and all of them will be called in some order.

\end{description}


%%% Local Variables: 
%%% mode: latex
%%% TeX-master: "manual1"
%%% End: 
