
\chapter{Programming with CDF}

\section{CDF as a Constraint Language}

As do most description logics, Type-1 CDF Ontologies have the ability
to express a large class of constraint problems in a succinct manner.

\begin{example} \rm
As a simple example of the relationship between ontologies and
constraints, we consider a simplified ontology of manufactured metal
alloys.  Metal alloys are characterized by a number of characteristics
such as chemical composition and the form in which the alloy is
primarily manufactured.  For example, the Americal Society for
Technical Manufacturers alloy ASTM B 107 is manufactured as a form
TUBE, WIRE, or ROD, while the Sikorsky Corporation specification SS
9705 indicates that the metal has the form BAR or TUBE.  This
information is obtained by the fragment.
%------------------------
{\tt  {\small 
\begin{tabbing} foo\=foo\=foo\=foo\=foo\=foo\=foooo\=foooooooooooooooo\=\kill
\> necessCond\_ext(cid('SS 9705',specs), \\
\> \> \>              vid(exists(rid(hasForm,specs),(cid(bar,specs) ; cid(tube,specs)) ), \\
\> \> \> 	              atMost(1,rid(hasForm,specs)) ) \\
\\
\> necessCond\_ext(cid('ASTM B 107',specs), \\
\>\> \>              vid(exists(rid(hasForm,specs),
	(cid(wire,specs) ; cid(tube,specs) ; cid(rod,specs)) ), \\
\\
\> necessCond\_ext(cid(bar,specs),vid(not(wire,specs))) \\
\> necessCond\_ext(cid(bar,specs),vid(not(tube,specs))) \\
\> necessCond\_ext(cid(bar,specs),vid(not(rod,specs))) \\
\> necessCond\_ext(cid(rod,specs),vid(not(tube,specs))) \\
\> necessCond\_ext(cid(rod,specs),vid(not(wire,specs))) \\
\> necessCond\_ext(cid(tube,specs),vid(not(wire,specs))) 
\end{tabbing} }}
%------------------------
CDF has been used in systems that read technical drawings of airplane
parts and infer properties about the parts.  If a given part {\tt
oid(p1,specs)} were described by both ASTM B 107 and SS 9705 the goal
{\tt
allModelsEntails(oid(p1,specs),exists(rid(hasForm,specs),cid(tube,specs)))}
would succeed, indicating that the part must have the form of a tube.
Such information could be useful in automatically deciding that the
part must be manufactured by a supplier with specialized tube-bending
machinery, which is often not available for suppliers who perform
general machining.
\end{example}

The next example is more abstract, and indicates how CDF ontologies
can represent propositional clauses.  As discussed in Section
\ref{sec:type1query}, in \version, the CDF theorem prover is not
especially efficient, and should not be used for

\begin{example} \rm
Consider the propositional clauses, $p \vee q \vee \neg r$ and $\neg p
\vee \neg q \vee r$.  These clauses can be represented by the ontology
%------------------------
{\tt  {\small 
\begin{tabbing} foo\=foo\=foo\=foo\=foo\=foo\=foooo\=foooooooooooooooo\=\kill
\> necessCond\_ext(oid(o1,prop), 
	(cid(p,prop) ; cid(q,prop) ; not(cid(r,prop))) ), \\
\> necessCond\_ext(oid(o1,prop), 
	(not(cid(p,prop)) ; not(cid(q,prop)) ; cid(r,prop)) ), \\
\end{tabbing} }}
%------------------------
The consistency of the two clauses can be determined by checking the
consistency of {\tt oid(o1,prop)}.
\end{example}

It is no surprise to a reader familiar with description logics that a
CDF ontology can represent even more expressive theories. A modal
formula, such as $\square p \vee \Diamond (q \wedge r)$ can easily be
expressed as a class expression

{\small {\tt 
all(rid(rel,modal),(cid(p,modal) ; exists(rid(rel,modal),(cid(q,modal) , cid(r,modal)))) } }

and incorporated into an ontology.  It is easy to see from
Chapter~\ref{chap:type1} that CDF class expressions express {\em
  multi-modal} formulas, in which more than one relation can exist
within a relational quantifier, and allows numeric constraints on
relational quantifiers as well.  On the other hand \version{} of CDF
does not allow specification of transitive relations~footnote{Soon!},
and so does not allow the representation of temporal logics such as
CTL* or the modal-$\mu$ calculus (see e.g. \cite{Stir94}).

\section{Non-Monotonic Reasoning and CDF}

XSB supports non-monotonic reasoning through its implementation of the
well-founded semantics, its support of ASP through the XASP package.
As a result, a number of non-monotonic formalisms have been
implemented in XSB and used for a variety of applications
(e.g.~\cite{Swif99a,CuSw02,AlPS03} However the use of negation within
CDF class expressions differs significantly from the use of negation
in XSB.  Consider an example taken from medical reasoning in which a
patient in the US with fever and headache is assumed to have influenza
unless other knowledge about the cause of the symptoms, say
menengitis, is present.  Such reasoning is easy to represent in
Prolog.

{\tt has\_influenza:- has\_fever, has\_headache, tnot(menengitis). }

If {\tt has\_fever}, and {\tt has\_headache} are both true, but
nothing is known about {\tt menengitis}, the atom {\tt has\_influenza}
will be inferred.  Consider a translation of the above rule to a class
expression: 

{\tt has\_influenza or not(has\_fever and  has\_headache and notmenengitis) }

Now if {\tt has\_fever}, and {\tt has\_headache} are both true the
above formula reduces to 

{\tt has\_influenza or  menengitis}.

So that either influenza or menengitis is consistent with the class
expression In other words, the preference for non-monotonically
inferring {\tt has\_infuenza} is not expressable with a simple class
expression.  Only a limited amount of work has been done to relate
reasoning in description logics to non-monotonic reasoning
(e.g. \cite{BaaH95}), and non-monotonic reasoning is not supported
within \version{} of CDF.  However, there is no restriction on using
non-monotonic negation within intentional rules.  Accordingly, the
above example could be written as
%
\begin{verbatim}
hasAttr_int(oid(Patient,ex),rid(hasDiagnosis,ex),cid(influenza,ex)):-
     has_fever,has_headache,tnot(menengitis).

hasAttr_int(oid(Patient,ex),rid(hasDiagnosis,ex),cid(menengitis,ex)):-
     menengitis.
\end{verbatim}
%
This example indicates a simple programming architecture for CDF.
Below and above CDF are any XSB rules that are used by intensional
rules, and these rules may use non-monotonic negation, numeric
constraints, or other features.  As long as, say, CDF intensional
rules do not depend non-monotonically on other CDF relations the
semantic properties of CDF will be preserved.  Of course, more
sophisticated notions of stratification of non-monotonic negation can
be explored, such as the ability of two disjoint CDF components to be
non-monotonically related.  In any case, \version{} of CDF does not
prohibit non-monotonic dependencies among CDF components, but a user
who programs such dependencies should fully understand the difference
between classical and non-monotonic negation and ensure that
non-monotonic dependencies do not compromise consistency and
entailment checking.

%-----------------------------------------------------------------------------------
\section{Using CDF Relations in Rules}

%-----------------------------------------------------------------------------------
\section{CDF and FLORA-2}

% f-logic arrows
\newcommand{\fd}{\ensuremath{{\rightarrow}}}                   % scalar
\newcommand{\bfd}{\ensuremath{{\bullet\!\!\!\fd}}}            % " + inheritable
\newcommand{\mvd}{\ensuremath{{\rightarrow\!\!\!\!\rightarrow}}}  % multivalued
\newcommand{\bmvd}{\ensuremath{{\bullet\!\!\!\mvd}}}              % " + inheritable
\newcommand{\Fd}{\ensuremath{{\Rightarrow}}}                      % scalar signature
\newcommand{\Mvd}{\ensuremath{{\Rightarrow\!\!\!\!\Rightarrow}}}  % multiv signature
\newcommand{\isa}{\,{\bf{:}}\,}
\newcommand{\subcl}{\,{\bf{::}}\,}

The Type-0 ontologies of CDF have an ``object-oriented'' or
frame-based flavor.  However, CDF is not the only object-oriented
packaged for XSB: the {\em FLORA-2} package \cite{YaKZ05} is also a
sophisticated package that allows object-oriented logic programming
based on the semantics of F-Logic \cite{KiLW95} and that has a growing
user community.  It is natural to ask about the relation between CDF
and F-Logic.  Figure~\ref{fig:flogic-model} (from \cite{YaKZ05}) is an introductory
example of an FLORA-2 object base concerning publications.  We explain
its semantics in terms of a translation into CDF
(Figure~\ref{fig:flogic-cdf}).  
%
%-----------------------------------------------------------------------------------
\begin{figure}[htb]
{\small 
\begin{tabular}{c}
  \begin{tabular}{l}
    {\bf Schema:}\\
    conf\_p\subcl paper. \\
    journal\_p\subcl paper.\\
    paper[authors\Mvd  person, title\Fd string].\\
    journal\_p[in\_vol\Fd journal\_vol]. \\
%    conf\_p[at\_conf\Fd conf\_proc].\\
    journal\_vol[of \Fd journal, volume\Fd integer, 
               year\Fd integer].\\  
    journal[name\Fd string, publisher\Fd string,
            editors\Mvd person]. \\
    person[name\Fd string, affil(integer)\Fd institution]. \\
    institution[name\Fd string, address\Fd string].\smallskip\\

    {\bf Objects:}\\
    $o_{j1}$\isa journal\_p[%
      title\fd 'Records, Relations, Sets, and Entities',
      authors\mvd$\{o_{mes}\}$, in\_vol\fd $o_{i11}$]. \\
    $o_{i11}$\isa journal\_vol[of\fd $o_{is}$, volume\fd 1, year\fd1975]. \\
    $o_{is}$\isa journal[name\fd'Information Systems', editors\mvd $\{o_{mj}\}$]. \\
    $o_{mes}$\isa person[name\fd'Michael E. Senko',affil(1976)\fd $o_{rwt}$]. \\
    $o_{rwt}$\isa institution[name\fd'RWTH\_Aachen'].
\end{tabular}
\end{tabular}
}
\caption{Part of a Publications Object Base and its Schema in {\em FLORA}-2}
  \label{fig:flogic-model}
\end{figure}
%-----------------------------------------------------------------------------------
%
From comparing the two figures, (or
the formal semantics of FLORA-2 and CDF) it can be seen that the
FLORA-2 schema operator {\tt ::/2} corresponds to an {\tt isa/2}
relation between two classes, while the object operator {\tt :/2}
corresponds to an {\tt isa/2} relation between an object and a class.
Other relations for classes and objects have a rather different syntax
in FLORA-2 than in CDF.  First note that in FLORA-2, the molecule {\tt
institution[name => string, address => string]} corresponds to two
simpler molecules {\tt institution[name => string]} and {\tt
institution[address => string]}.  Each of these latter molecules can
be translated into 2 CDF facts, an {\tt allAttr\_ext/3} fact to denote
the typing and a {\tt maxAttr/4} fact to denote the cardinality.  The
FLORA-2 operator {\tt =>>} indicates a non-functional schema
dependency and can be translated using an {\tt allAttr\_ext/3} fact
alone.  At the object level, the operators {\tt ->} and {\tt ->>} are
each handled by a {\tt hasAttr\_ext} fact for the appropriate object:
information that the dependency is functional in a {\tt -> } relation
is handled by the CDF schema and does not need to be repeated.  Also
note that the FLORA-2 {\tt @} operator, which is used to handle the
non-binary (i.e. parameterized) attribute {\tt affil} for a person is
reflected by the CDF product class.

FLORA-2 syntax is more succinct for this example than CDF syntax for
three reasons: the use of F-Logic molecules in FLORA-2, the use of
components in CDF identifiers, and the need for both a {\tt
allAttr\_ext/3} and a {\tt maxAttr/4} fact in the translation of each
{\tt =>} operator.  These features stem from the different design
choices behind CDF and FLORA-2.  FLORA-2 is intended for programmers
who want to program in an object-oriented logic, while CDF is intended
as a Prolog-based repository for logically structured knowledge.  From
this latter perspective the use of components in CDF means that
objects can have more meaningful identifiers than shown in
Figure~\ref{fig:flogic-model} -- {\tt oid('Michael E. Senko',flora)}
would have a different meaning and different properties than a
'Michael E. Senko' taken from a different ontology and having a
different component.  Furthermore, the necessity of using an {\tt
allAttr/3} along with a {\tt maxAttr/3} relation to represent {\tt =>
} arises from the ability of CDF to represent fine shades of meaning
within the schema of an ontology.  For instance, it may be the case
that a heterosexual male has many lovers, all of whom are women, but
that only one of the women is a primary lover.  It is difficult to
express this meaning in FLORA-2, but it can be represented in CDF as

{\tt 
\begin{tabular}{l}
isa\_ext(rid(hasPrimaryLover,ex),rid(hasLover,ex)). \\
allAttr\_ext(cid(heteroMale,ex),rid(hasLover,ex),cid(woman,ex)). \\
maxAttr\_ext(cid(heteroMale,ex),rid(hasPrimaryLover,Ex),cid(woman,ex)).
\\
\end{tabular}
}

\noindent
The above differences, however, are relatively minor compared to the
main differences between the systems.  CDF supports consistency and
entailment checking for Type-1 ontologies which FLORA-2 does not
support; FLORA-2 however has support for non-monotonic inheritance
which is not supported in CDF (nor in description logics).  By and
large a determined programmer could obtain most of the benefits of
FLORA-2 using a combination of CDF and XSB, while a determined FLORA-2
programmer could obtain most of the benefits of CDF within FLORA-2
(perhaps even by translating monotonic FLORA-2 knowledge bases to
class expressions and calling the CDF theorem prover).

%-----------------------------------------------------------------------------------
\begin{figure}[htb]
{\small
\begin{tabular}{c}
  \begin{tabular}{l}
    {\bf Schema:}\\
    isa\_ext(cid(conf\_p,flora),cid(paper,flora)). \\
    isa\_ext(cid(journal\_p,flora),cid(paper,flora)). \\
\ \\
%    hasAttr\_ext(cid(paper,flora),rid(authors,flora),cid(person,flora)). \\
    allAttr\_ext(cid(paper,flora),rid(authors,flora),cid(person,flora)).  \\
%    hasAttr\_ext(cid(paper,flora),rid(title,flora),cid(allAtoms,cdfpt)). \\
    allAttr\_ext(cid(paper,flora),rid(title,flora),cid(allAtoms,cdfpt)).  \\
    maxAttr\_ext(cid(paper,flora),rid(title,flora),cid(allAtoms,cdfpt)). \\
\ \\    
%    hasAttr\_ext(cid(journal\_p,flora),rid(in\_vol,flora),cid(journal\_volume,flora)). \\
    allAttr\_ext(cid(journal\_p,flora),rid(in\_vol,flora),cid(journal\_volume,flora)).  \\
    maxAttr\_ext(cid(journal\_p,flora),rid(in\_vol,flora),cid(journal\_volume,flora)).  \\
\ \\
    allAttr\_ext(cid(journal\_vol,flora),rid(of,flora),cid(journal,flora)). \\
    maxAttr\_ext(1,cid(journal\_vol,flora),rid(of,flora),cid(journal,flora)). \\
    allAttr\_ext(cid(journal\_vol,flora),rid(volume,flora),cid(allIntegers,cdfpt)). \\
    maxAttr\_ext(1,cid(journal\_vol,flora),rid(volume,flora),cid(allIntegers,cdfpt)). \\
%    allAttr\_ext(cid(journal\_vol,flora),rid(number,flora),cid(allIntegers,cdfpt)).\\
%    maxAttr\_ext(1,cid(journal\_vol,flora),rid(number,flora),cid(allIntegers,cdfpt)).\\
    allAttr\_ext(cid(journal\_vol,flora),rid(year,flora),cid(allIntegers,cdfpt)).\\
    maxAttr\_ext(1,cid(journal\_vol,flora),rid(year,flora),cid(allIntegers,cdfpt)).\\
\ \\
    allAttr\_ext(cid(journal,flora),rid(name,flora),cid(allAtoms,cdfpt)). \\
    maxAttr\_ext(1,cid(journal,flora),rid(name,flora),cid(allAtoms,cdfpt)). \\
    allAttr\_ext(cid(journal,flora),rid(publisher,flora),cid(allAtoms,cdfpt)). \\
    maxAttr\_ext(1,cid(journal,flora),rid(publisher,flora),cid(allAtoms,cdfpt)). \\
    allAttr\_ext(cid(journal,flora),rid(editors,flora),cid(person,flora)). \\
\ \\
    allAttr\_ext(cid(person,flora),rid(name,flora),cid(allAtoms,cdfpt)). \\
    maxAttr\_ext(1,cid(person,flora),rid(name,flora),cid(allAtoms,cdfpt)). \\
    allAttr\_ext(cid(person,flora),rid(affil,flora),cid(pair(cid(institution,flora),cid(allIntegers,cdfpt)))). \\
    maxAttr\_ext(1,cid(person,flora),rid(affil,flora),cid(pair(cid(institution,flora),cid(allIntegers,cdfpt)))). \\
\ \\
    allAttr\_ext(cid(institution,flora),rid(name,flora),cid(allAtoms,cdfpt)). \\
    maxAttr\_ext(1,cid(institution,flora),rid(name,flora),cid(allAtoms,cdfpt)). \\
    allAttr\_ext(cid(institution,flora),rid(address,flora),cid(allAtoms,cdfpt)). \\
    maxAttr\_ext(1,cid(institution,flora),rid(address,flora),cid(allAtoms,cdfpt)). \\
\ \\    
    {\bf Objects:}\\
    isa\_ext(oid($o_{j1}$,flora),cid(journal,flora)). \\
    hasAttr\_ext(oid($o_{j1}$,flora),rid(title,flora),cid('Records, Relations, Sets, and Entities',cdfpt)). \\
    hasAttr\_ext(oid($o_{j1}$,flora),rid(authors,flora),oid($o_{mes}$,flora)). \\
    hasAttr\_ext(oid($o_{j1}$,flora),rid(in\_vol,flora),oid($o_{i11}$,flora)). \\
\ \\
    hasAttr\_ext(oid($o_{i11}$,flora),rid(of,flora),oid($o_{is}$,flora)) \\
%    hasAttr\_ext(oid($o_{i11}$,flora),rid(number,flora),oid(1,cdfpt)) \\
    hasAttr\_ext(oid($o_{i11}$,flora),rid(volume,flora),oid(1,cdfpt)) \\
    hasAttr\_ext(oid($o_{i11}$,flora),rid(year,flora),oid(1976,cdfpt)) \\
\ \\
    hasAttr\_ext(oid($o_{mes}$,flora),rid(name,flora),
	oid('Michael E. Senko',cdfpt)). \\
    hasAttr\_ext(oid($o_{mes}$,flora),rid(affil,flora),
	oid(pair(oid($o_{rwt}$,flora),oid(1976,cdfpt)). \\
\ \\
    hasAttr\_ext(oid($o_{rwt}$,flora),rid(name,flora),
	oid('RWTH\_Aachen',cdfpt)). 
\end{tabular}
\end{tabular}
}
\caption{CDF Encoding for FLORA-2}
  \label{fig:flogic-cdf}
\end{figure}
%-----------------------------------------------------------------------------------


% equality.
